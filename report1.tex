\documentclass{jsarticle}
\usepackage{listings}
\begin{document}

\title{計算モデル論演習Ⅱレポート}
\subtitle{Nクイーン問題について}
\auther{平松 亨隆}
\maketitle

\section{課題1-1}
\subsection{探索による解法の説明}
Nクイーン問題とは、4以上である任意の数NのクイーンをN×Nのマスに配置する問題である。しかし、すべてのクイーンは縦・横・斜めのどの方向にもクイーンが存在してはいけない、という制約がある。

この問題の探索による解法をN=4の場合で説明する。前提条件として各列各行には一つづつクイーンが存在するはずである。今回は列を基準にして左側から順番にクイーンを配置していくことにする。

まず、何もクイーンは配置されていないので一列目の一行目にクイーンを配置する。その次の二列目では一行目と二行目はそれぞれ横と斜めに一列目のクイーンがあるため、三行目にクイーンを設置する。しかし、三列目では一行目三行目が一列目のクイーンと干渉し、二行目四行目は二列目のクイーンと干渉する。よってクイーンを設置できないため、二列目に戻る。そこで四行目にも配置できるため、二列目のクイーンを四行目に配置する。三列目は一行目が一列目と干渉するため、二行目に配置する。しかし次の四列目ではどの行も他のクイーンと干渉し、三列目も二行目以外は干渉、二列目は四行目に配置されていることから再配置が出来ないため、一列目の再配置を行う。一列目のクイーンを一行目から二行目に再配置。次に二列目は一、二、三行目が干渉するため四行目に配置。その次の三列目は一行目に配置。最後に四行目も一行目二行目が干渉するため、三行目に配置。ここでようやくすべてのクイーンを互いに干渉しない位置に配置できた。

このように少しづつクイーンの位置をずらしていけば必ず正しいクイーンの配置調べることができる。

\subsection{組合せ最適化問題とは?}



2
変数一覧
関数の意味

3
できるだけお大きなN

4

\section{課題1-2}
\subsection{関数と変数の説明}
\subsubsection{print_queens関数}
この関数はクイーンの配置を'●'で、何も置かれていないマスを'□'で表してn×nの盤を出力するvoid型の関数である。
引数は、クイーンの位置情報のポインタQと盤のサイズを示すint型のn。(下記プログラムの行番号6)
この関数で宣言される変数は2つ。現在いる盤の列を示す変数xと現在いる盤の行を示す変数yである。(下記プログラムの行番号7)

この関数は、変数xとyと使って二重のfor文を作り、引数として与えられたポインタQと盤のサイズnから盤全体を探索する。その過程で任意のマスに1(クイーンがおいてあった場合の数字)があった場合は'●'を、それ以外があった場合は'□'を出力する。よって最後まで探索を終える頃には'●'と'□'を使ったn×nの盤が出力される。(下記プログラムの行番号8~16)

\subsubsection{check_queens関数}
この関数は現在いる位置にクイーンをおいた場合、他のクイーンと干渉するかどうかを調べるint型の関数である。
引数は現在いる場所の列を示す

\begin{lstlisting}[basicstyle=\ttfamily\footnotesize, frame=single]

     1	#include <stdio.h>
     2	#include <stdlib.h>
     3	#define N 30
     4	
     5	
     6	void print_queens(int *Q,int n){
     7	  int x, y;
     8	  for(x=0; x<n; x++){
     9	    for(y=0; y<n; y++){
    10	      if(Q[x*n+y] == 1)
    11		printf("● ");
    12	      else
    13		printf("□ ");
    14	    }
    15	    printf("\n");
    16	  }
    17	}
    18	
    19	
    20	int check_queens(int i, int j, int *Q, int n){
    21	  int x, y;
    22	  int sum=0;
    23	  for(x=0; n>x; x++){
    24	    for(y=0; n>y; y++){
    25	      if(x==i && y==j){
    26	      }
    27	      else if(x==i && Q[x*n+y]==1){
    28		sum++;
    29	      }
    30	      else if(y==j && Q[x*n+y]==1){
    31		sum++;
    32	      }
    33	      else if(i+j==x+y && Q[x*n+y]==1){
    34		sum++;
    35	      }
    36	      else if(i-j==x-y && Q[x*n+y]==1){
    37		sum++;
    38	      }
    39	    }
    40	  }
    41	  return sum;
    42	}
    43	
    44	
    45	
    46	int main (void){
    47	  int *Q, P[N]={0}, n;
    48	  int x, y,c,f=0,count=0,S[N]={0};
    49	  printf("Number of queen:");
    50	  scanf("%d",&n);
    51	  Q=calloc(n*n,sizeof(int));
    52	  for(x=0;x<n;x++){
    53	    for(y=P[x];y<n;y++){
    54	      c=check_queens(x,y,Q,n);
    55	      if(c==0){
    56	      if(f==0){
    57		count++;
    58		f=1;
    59	      }
    60		Q[x*n+y]=1;
    61		P[x]=y;
    62	      }
    63	      S[x]=S[x]+Q[x*n+y];
    64	    }
    65	    if(S[x]==0){
    66	      f=0;
    67	      P[x]=0;
    68	      y=P[--x];
    69	      Q[x*n+y]=0;
    70	      P[x]++;
    71	      x=x-1;
    72	      S[x+1]=0;
    73	    }
    74	  }
    75	  printf("success\n");
    76	  print_queens(Q,n);
    77	  printf("Number of trials:%d\n",count);
    78	  free(Q);
    79	  return 0;
    80	}


 \end{lstlisting}


\end{document}
