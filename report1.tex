\documentclass{jsarticle}
\usepackage{listings}
\begin{document}

\title{計算モデル論演習Ⅱレポート}
\subtitle{Nクイーン問題について}
\auther{平松 亨隆}
\maketitle

\section{課題1-1}
\subsection{探索による解法の説明}
Nクイーン問題とは、4以上である任意の数NのクイーンをN×Nのマスに配置する問題である。しかし、すべてのクイーンは縦・横・斜めのどの方向にもクイーンが存在してはいけない、という制約がある。

この問題の探索による解法を説明する。まず、前提条件として各列各行には一つづつクイーンが存在するはずである。今回は列を基準にしてクイーンを配置していくことにする。




\begin{lstlisting}[basicstyle=\ttfamily\footnotesize, frame=single]

     1	#include <stdio.h>
     2	#include <stdlib.h>
     3	#define N 30
     4	
     5	
     6	void print_queens(int *Q,int n){
     7	  int x, y;
     8	  for(x=0; x<n; x++){
     9	    for(y=0; y<n; y++){
    10	      if(Q[x*n+y] == 1)
    11		printf("● ");
    12	      else
    13		printf("□ ");
    14	    }
    15	    printf("\n");
    16	  }
    17	}
    18	
    19	
    20	int check_queens(int i, int j, int *Q, int n){
    21	  int x, y;
    22	  int sum=0;
    23	  for(x=0; n>x; x++){
    24	    for(y=0; n>y; y++){
    25	      if(x==i && y==j){
    26	      }
    27	      else if(x==i && Q[x*n+y]==1){
    28		sum++;
    29	      }
    30	      else if(y==j && Q[x*n+y]==1){
    31		sum++;
    32	      }
    33	      else if(i+j==x+y && Q[x*n+y]==1){
    34		sum++;
    35	      }
    36	      else if(i-j==x-y && Q[x*n+y]==1){
    37		sum++;
    38	      }
    39	    }
    40	  }
    41	  return sum;
    42	}
    43	
    44	
    45	
    46	int main (void){
    47	  int *Q, P[N]={0}, n;
    48	  int x, y,c,f=0,count=0,S[N]={0};
    49	  printf("Number of queen:");
    50	  scanf("%d",&n);
    51	  Q=calloc(n*n,sizeof(int));
    52	  for(x=0;x<n;x++){
    53	    for(y=P[x];y<n;y++){
    54	      c=check_queens(x,y,Q,n);
    55	      if(c==0){
    56	      if(f==0){
    57		count++;
    58		f=1;
    59	      }
    60		Q[x*n+y]=1;
    61		P[x]=y;
    62	      }
    63	      S[x]=S[x]+Q[x*n+y];
    64	    }
    65	    if(S[x]==0){
    66	      f=0;
    67	      P[x]=0;
    68	      y=P[--x];
    69	      Q[x*n+y]=0;
    70	      P[x]++;
    71	      x=x-1;
    72	      S[x+1]=0;
    73	    }
    74	  }
    75	  printf("success\n");
    76	  print_queens(Q,n);
    77	  printf("Number of trials:%d\n",count);
    78	  free(Q);
    79	  return 0;
    80	}


 \end{lstlisting}


\end{document}
